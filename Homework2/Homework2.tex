%%%%%%%%%%%%%%%%%%%%%%%%%%%%% Define Article %%%%%%%%%%%%%%%%%%%%%%%%%%%%%%%%%%
\documentclass{article}
%%%%%%%%%%%%%%%%%%%%%%%%%%%%%%%%%%%%%%%%%%%%%%%%%%%%%%%%%%%%%%%%%%%%%%%%%%%%%%%

%%%%%%%%%%%%%%%%%%%%%%%%%%%%% Using Packages %%%%%%%%%%%%%%%%%%%%%%%%%%%%%%%%%%
\usepackage{geometry}
\usepackage{graphicx}
\usepackage{amssymb}
\usepackage{amsmath}
\usepackage{amsthm}
\usepackage{empheq}
\usepackage{mdframed}
\usepackage{booktabs}
\usepackage{lipsum}
\usepackage{graphicx}
\usepackage{color}
\usepackage{psfrag}
\usepackage{pgfplots}
\usepackage{bm}
\usepackage{bookmark}
\usepackage{indentfirst}
\usepackage{enumitem}
%%%%%%%%%%%%%%%%%%%%%%%%%%%%%%%%%%%%%%%%%%%%%%%%%%%%%%%%%%%%%%%%%%%%%%%%%%%%%%%

% Other Settings

%%%%%%%%%%%%%%%%%%%%%%%%%% Page Setting %%%%%%%%%%%%%%%%%%%%%%%%%%%%%%%%%%%%%%%
\geometry{a4paper}

%%%%%%%%%%%%%%%%%%%%%%%%%% Define some useful colors %%%%%%%%%%%%%%%%%%%%%%%%%%
\definecolor{ocre}{RGB}{243,102,25}
\definecolor{mygray}{RGB}{243,243,244}
\definecolor{deepGreen}{RGB}{26,111,0}
\definecolor{shallowGreen}{RGB}{235,255,255}
\definecolor{deepBlue}{RGB}{61,124,222}
\definecolor{shallowBlue}{RGB}{235,249,255}
%%%%%%%%%%%%%%%%%%%%%%%%%%%%%%%%%%%%%%%%%%%%%%%%%%%%%%%%%%%%%%%%%%%%%%%%%%%%%%%

%%%%%%%%%%%%%%%%%%%%%%%%%% Define an orangebox command %%%%%%%%%%%%%%%%%%%%%%%%
\newcommand\orangebox[1]{\fcolorbox{ocre}{mygray}{\hspace{1em}#1\hspace{1em}}}
%%%%%%%%%%%%%%%%%%%%%%%%%%%%%%%%%%%%%%%%%%%%%%%%%%%%%%%%%%%%%%%%%%%%%%%%%%%%%%%

%%%%%%%%%%%%%%%%%%%%%%%%%%%% English Environments %%%%%%%%%%%%%%%%%%%%%%%%%%%%%
\newtheoremstyle{mytheoremstyle}{3pt}{3pt}{\normalfont}{0cm}{\rmfamily\bfseries}{}{1em}{{\color{black}\thmname{#1}~\thmnumber{#2}}\thmnote{\,--\,#3}}
\newtheoremstyle{myproblemstyle}{3pt}{3pt}{\normalfont}{0cm}{\rmfamily\bfseries}{}{1em}{{\color{black}\thmname{#1}~\thmnumber{#2}}\thmnote{\,--\,#3}}
\theoremstyle{mytheoremstyle}
\newmdtheoremenv[linewidth=1pt,backgroundcolor=shallowGreen,linecolor=deepGreen,leftmargin=0pt,innerleftmargin=20pt,innerrightmargin=20pt,]{theorem}{Theorem}[section]
\theoremstyle{mytheoremstyle}
\newmdtheoremenv[linewidth=1pt,backgroundcolor=shallowBlue,linecolor=deepBlue,leftmargin=0pt,innerleftmargin=20pt,innerrightmargin=20pt,]{definition}{Definition}[section]
\theoremstyle{myproblemstyle}
\newmdtheoremenv[linecolor=black,leftmargin=0pt,innerleftmargin=10pt,innerrightmargin=10pt,]{problem}{Problem}[section]
%%%%%%%%%%%%%%%%%%%%%%%%%%%%%%%%%%%%%%%%%%%%%%%%%%%%%%%%%%%%%%%%%%%%%%%%%%%%%%%

%%%%%%%%%%%%%%%%%%%%%%%%%%%%%%% Plotting Settings %%%%%%%%%%%%%%%%%%%%%%%%%%%%%
\usepgfplotslibrary{colorbrewer}
\pgfplotsset{width=8cm,compat=1.9}
%%%%%%%%%%%%%%%%%%%%%%%%%%%%%%%%%%%%%%%%%%%%%%%%%%%%%%%%%%%%%%%%%%%%%%%%%%%%%%%

%%%%%%%%%%%%%%%%%%%%%%%%%%%%%%% Title & Author %%%%%%%%%%%%%%%%%%%%%%%%%%%%%%%%
\title{Homework 2}
\author{Nguyen Tuan Anh - 2252038 - CN01}
%%%%%%%%%%%%%%%%%%%%%%%%%%%%%%%%%%%%%%%%%%%%%%%%%%%%%%%%%%%%%%%%%%%%%%%%%%%%%%%

\begin{document}
    \maketitle
    \section*{Section 2.1}
    \subsection*{Exercise 11}
        Determine whether each of these statements is true or false.
        \begin{enumerate} [label = (\alph*)]
            \item 0 $ \in \emptyset$
            \item $ \emptyset \in \{0\}$
            \item $ \{0\} \subset \emptyset $
            \item $ \emptyset \subset \{0\} $
            \item $ \{0\} \in \{0\} $
            \item $ \{0\} \subset \{0\} $
            \item $ \{\emptyset\} \subseteq \{\emptyset\}$
        \end{enumerate}
    \subsubsection*{Solution}
        \begin{enumerate} [label = (\alph*)]
            \item 0 $ \in \emptyset$: This statement is false because $\emptyset$
            has no elements so 0 can not be an element of the empty set.
            \item $ \emptyset \in \{0\}$: This statement is false because
            $ \emptyset $ is not an element in set \{0\}.
            \item $ \{0\} \subset \emptyset $: This statement is false because
            $ \emptyset $ has no elements so that \{0\} can not be a subset
            of $ \emptyset $.
            \item $ \emptyset \subset \{0\} $: This statement is true because
            $ \emptyset $ is one of the two sets that every nonempty set is
            guaranteed to have.
            \item $ \{0\} \in \{0\} $: This statement is false because 
            $ \{0\} $ is an element of \{\{0\}\} not an element of \{0\}.
            \item $ \{0\} \subset \{0\} $: This statement is false because the
            two set all have the same elements 0 so that is must be $ \subseteq $.
            \item $ \{\emptyset\} \subseteq \{\emptyset\}$: This statement is true
            because both singleton set have the same element $ \emptyset $. Therefore, this
            statement is true.
        \end{enumerate}
    \subsection*{Exercise 12}
        Determine whether these statements are true or false.
        \begin{enumerate} [label = (\alph*)]
            \item $ \emptyset \in \{\emptyset\} $
            \item $ \emptyset \in \{\emptyset,\{\emptyset\}\} $
            \item $ \{\emptyset\} \in \{\emptyset\} $
            \item $ \{\emptyset\} \in \{\{\emptyset\}\} $
            \item $ \{\emptyset\} \subset \{\emptyset, \{\emptyset\}\} $
            \item $ \{\{\emptyset\}\} \subset \{\emptyset, \{\emptyset\}\} $
            \item $ \{\{\emptyset\}\} \subset \{\{\emptyset\}, \{\emptyset\}\} $
        \end{enumerate}
    \subsubsection*{Solution}
        \begin{enumerate} [label = (\alph*)]
            \item \(\emptyset \in \{\emptyset\}\): This statement is true
            because \(\emptyset\) is an element of a singleton set contains 
            element \(\emptyset\).
            \item \(\emptyset \in \{\emptyset,\{\emptyset\}\}\): This statement
            is true because \(\emptyset\) is an element of the set 
            $\{\emptyset,\{\emptyset\}\}$.
            \item $ \{\emptyset\} \in \{\emptyset\} $: This statement is false
            because \(\emptyset\) must be an element of \(\{\{\emptyset\}\}\).
            \item $ \{\emptyset\} \in \{\{\emptyset\}\} $: This statement is true
            because the set \(\{\{\emptyset\}\}\) contains \(\{\emptyset\}\).
            \item $ \{\emptyset\} \subset \{\emptyset, \{\emptyset\}\} $: This
            statement is true because \(\emptyset\) is an element is the set \(\{\emptyset, \{\emptyset\}\}\)
            so the set contains \(\emptyset\) is a subset of $\{\emptyset, \{\emptyset\}\}$.
            \item $ \{\{\emptyset\}\} \subset \{\emptyset, \{\emptyset\}\} $: This
            statement is true and its reason is the same as problem (e).
            \item $ \{\{\emptyset\}\} \subset \{\{\emptyset\}, \{\emptyset\}\} $:
            We can see that the set $\{\{\emptyset\}, \{\emptyset\}\}$ has two elements which
            are equal to each other. Therefore, we can simplify it to \(\{\{\emptyset\}\}\).
            Therefore, this statement is false because these sets are equal to each other so
            it must be \(\subseteq\) instead of \(\subset\).
        \end{enumerate}
    \subsection*{Exercise 13}
        Determine whether each of these statements is true or false.
        \begin{enumerate} [label = (\alph*)]
            \item \(x \in \{x\}\)
            \item \(\{x\} \subseteq \{x\}\)
            \item \(\{x\} \in \{x\}\)
            \item \(\{x\} \in \{\{x\}\}\)
            \item \(\emptyset \subseteq \{x\}\)
            \item \(\emptyset \in \{x\}\)
        \end{enumerate}
    \subsubsection*{Solution}
        \begin{enumerate} [label = (\alph*)]
            \item \(x \in \{x\}\): This statement is true because \(x\) is an element in set
            \(x\).
            \item \(\{x\} \subseteq \{x\}\): This statement is true.
            \item \(\{x\} \in \{x\}\): This statement is false because \(x\) is an element of 
            \(\{\{x\}\}\) not \(\{x\}\).
            \item \(\{x\} \in \{\{x\}\}\): This statement is true due to the reason from problem(c).
            \item \(\emptyset \subseteq \{x\}\): This statement is true according to \textbf{Theorem 1}.
            \item \(\emptyset \in \{x\}\): This statement is false because \(\emptyset\) is not 
            an element of set \(\{x\}\).
        \end{enumerate}
    \subsection*{Exercise 26}
        Determine whether each of these sets is the power set of 
        a set, where a and b are distinct elements.
        \begin{enumerate} [label = (\alph*)]
            \item \(\emptyset\)
            \item \(\{\emptyset, \{a\}\}\)
            \item \(\{\emptyset, \{a\}, \{\emptyset, a\}\}\)
            \item \(\{\emptyset, \{a\}, \{b\}, \{a, b\}\}\)
        \end{enumerate}
    \subsubsection*{Solution}
        The set(d) is the power set of set \{a, b\} because the set has two elements a and b
        so that its power set has \(2^2 = 4\) elements, which has the same number of elements
        of set(d). 
        \begin{align*}
            \mathcal{P}(\{a, b\}) = \{\emptyset, \{a\}, \{b\}, \{a, b\}\}
        \end{align*}
    \subsection*{Exercise 27}
        Prove that \(\mathcal{P}(A) \subseteq \mathcal{P}(B)\) if and only if \(A\subseteq B\).
    \subsubsection*{Solution}
        There are two things we need to prove:
        \begin{align*}
            (\mathcal{P}(A) \subseteq \mathcal{P}(B) \to A \subseteq B) \land (A \subseteq B \to \mathcal{P}(A) \subseteq \mathcal{P}(B))
        \end{align*}
        \begin{itemize}
            \item \(\mathcal{P}(A) \subseteq \mathcal{P}(B) \to A \subseteq B\):\\
            
            \(\mathcal{P}(A) \subseteq \mathcal{P}(B)\) means that every element of power set A
            is also an element of power set B. Addtionally, we all know that power set of a set
            has \(2^n\) elements created from the combinations of all the elements from the original
            set. Because all every element of power set A is also an element of power set B so we can infer that
            the element of set A is also an element of set B because they have the same combinations in the power set.
            Therefore, this case is true.
            \item \(A \subseteq B \to \mathcal{P}(A) \subseteq \mathcal{P}(B)\)\\
            
            \(A \subseteq B\) means that every element of A is also an element of B. Because A and B
            have the same element so that there combinations of elements of these two sets will be the same.
            Therefore, the elements power set of A and B will be the same because both of them contain
            all subsets of A and B(We have that \(A \subseteq B\)). Therefore, this case is true.
        \end{itemize}
        Because both cases are true so that \((\mathcal{P}(A) \subseteq \mathcal{P}(B) \to A \subseteq B) \land (A \subseteq B \to \mathcal{P}(A) \subseteq \mathcal{P}(B))\)
        is true and it is equivalent to \(\mathcal{P}(A) \subseteq \mathcal{P}(B) \leftrightarrow A \subseteq B\).
    \subsection*{Exercise 28} Show that if \(A \subseteq C\) and \(B \subseteq D\), then \(A \times B \subseteq C \times D\)
    \subsubsection*{Solution} Because A is a subset of C and B is a subset of D. Suppose that we have set A, B, C and D:
        \begin{itemize}
            \item \(A = \{a, b\}\)
            \item \(B = \{c, d\}\)
            \item \(C = \{a, b\}\)
            \item \(D = \{c, d\}\)
        \end{itemize}
        We have that: 
        \begin{align*}
            A \times B = \{(a,c),(a,d),(b,c),(b,d)\}\\
            C \times D = \{(a,c),(a,d),(b,c),(b,d)\}
        \end{align*}
        After using \textbf{Cartesian Product} to calculate \(A \times B\), \(C \times D\), we can
        see that every element \(A \times B\) is also the element of \(C \times D\). Therefore,
        we can infer that \(A \times B \subseteq C \times D\).
    \subsection*{Exercise 41}
        Explain why $A \times B \times C$ and $(A \times B) \times C$ are not the same.
    \subsubsection*{Solution}
        Let \(A = \{0, 1\}, B = \{1, 2\}, C = \{0, 1, 2\}\)
        \begin{align*}
            A \times B \times C &= \{(0, 1, 0), (0, 1, 1), (0, 1, 2), (0, 2, 0), (0, 2, 1),\\
            &(0, 2, 2),(1, 1, 0), (1, 1, 1), (1, 1, 2), (1, 2, 0), (1, 2, 1), (1, 2, 2)\}\\
            A \times B &= \{(0,1), (0,2), (1,1), (1,2)\}\\
            (A \times B) \times C &= \{((0,1),0,),((0, 1), 1), ((0, 1), 2), ((0, 2), 0), ((0, 2), 1),\\
            &((0, 2), 2), ((1, 1), 0), ((1, 1), 1), ((1, 1), 2), ((1, 2), 0), ((1, 2), 1), ((1, 2), 2)\}
        \end{align*}
        As we can see that \(A \times B \times C\) gives us  a set of 3-tuples has a form (a, b, c) with 
        \(a \in A, b \in B, c \in C\). However, \((A \times B) \times C\) gives us a set of 2-tuples has
        a form ((a, b), c) with \((a,b) \in A \times B\) and \(c \in C\). This is different from \(A \times B \times C\)
        because \(A \times B \times C\) is a 3-tuples but \((A \times B) \times C\) is a 2-tuples which has the
        first element is a ordered pair of \(A \times B\). Therefore, \(A \times B \times C\) is different from \((A \times B) \times C\)
    \subsection*{Exercise 42}
        Explain why\((A\times B)\times(C\times D)\)and \(A\times(B\times C)\times D\) are not the same.
    \subsubsection*{Solution}
        Let a, b, c, d are elements of set A, B, C, D respectively. Therefore, we have that
        \(a \in A, b \in B, c \in C, d \in D\). We get that \(A \times B\) will be a set
        consists ordered pairs (a, b) and \(C \times D\) also consists ordered pair (c, d).
        Because (a,b) and (c,d) is an element of set \(A \times B\) and \(C \times D\).
        Therefore, if we do the Cartesian product \((A\times B)\times (C\times D)\). The result
        will be a new set consists ordered pair ((a,b),(c,d)) which has the two elements are
        two ordered pairs from \(A \times B\) and \(C \times D\).\\
        
        However, \(A \times (B\times C) \times D\) is completely different. We get \(B \times C\) will be a set consists ordered
        pair (b, c) and continue to calculate \(A \times (B \times C) \times D\), we will get a new set consists 3-tuples
        (a,(b,c),d) with the fist element is an element \(\in A\), the second element is the element \(\in (B \times C)\), and
        the last element is an element \(\in D\).\\

        Therefore, we can infer that \((A \times B)\times(C \times D)\) and \(A \times (B \times C)\times D\) are different.
    \subsection*{Exercise 43}
        Prove or disprove that if A and B are sets, then \(\mathcal{P}(A \times B) = \mathcal{P}(A) \times \mathcal{P}(B)\)
    \subsubsection*{Solution}
        Let \(A = \{0, 1\}\) and \(B = {1, 2}\). Therefore, we get that \(|A| = |B| = 2\) and \(|A \times B| = |A| \times |B| = 2 \times 2 = 4\).
        Because \(A \times B\) has 4 elements so that \(\mathcal{P}(A)\) will have \(2^{|A|\times|B|} = 2^4 = 16\). However, we get that
        \(|\mathcal{P}(A)| = 2^{|A|} = 2^2 = 4\) and \(|\mathcal{P}(B)| = 2^{|B|} = 2^2 = 4\) and
        \begin{align*}
            |\mathcal{P}(A) \times \mathcal{P}(B)| = 2^{|A|}\times 2^{|B|} = 2^{|A| + |B|} = 2^4 =16
        \end{align*}
        Although the result of \(|\mathcal{P}(A\times B)|\) and \(|\mathcal{P}(A) \times \mathcal{P}(B)|\) are both equal to
        16 but the way they give the result 16 are completely different.
        \begin{align*}
            |\mathcal{P}(A\times B)| = 2^{|A| \times |B|} = 2^4 = 16\\
            |\mathcal{P}(A) \times \mathcal{P}(B)| = 2^{|A| + |B|} = 16
        \end{align*}
        Because \(2^{|A| \times |B|} \neq 2^{|A| + |B|}\) so we can conclude that \(\mathcal{P}(A \times B) \neq \mathcal{P}(A) \times \mathcal{P}(B)\)
    \subsection*{Exercise 44}
        Prove or disprove that if A, B, and C are nonempty sets and \(A \times B = A \times C\), then \(B = C\).
\end{document}