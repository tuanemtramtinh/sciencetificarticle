%%%%%%%%%%%%%%%%%%%%%%%%%%%%% Define Article %%%%%%%%%%%%%%%%%%%%%%%%%%%%%%%%%%
\documentclass{article}
%%%%%%%%%%%%%%%%%%%%%%%%%%%%%%%%%%%%%%%%%%%%%%%%%%%%%%%%%%%%%%%%%%%%%%%%%%%%%%%

%%%%%%%%%%%%%%%%%%%%%%%%%%%%% Using Packages %%%%%%%%%%%%%%%%%%%%%%%%%%%%%%%%%%
\usepackage{geometry}
\usepackage{graphicx}
\usepackage{amssymb}
\usepackage{amsmath}
\usepackage{amsthm}
\usepackage{empheq}
\usepackage{mdframed}
\usepackage{booktabs}
\usepackage{lipsum}
\usepackage{graphicx}
\usepackage{color}
\usepackage{psfrag}
\usepackage{pgfplots}
\usepackage{bm}
\usepackage{bookmark}
\usepackage{indentfirst}
\usepackage{enumitem}
%%%%%%%%%%%%%%%%%%%%%%%%%%%%%%%%%%%%%%%%%%%%%%%%%%%%%%%%%%%%%%%%%%%%%%%%%%%%%%%

% Other Settings

%%%%%%%%%%%%%%%%%%%%%%%%%% Page Setting %%%%%%%%%%%%%%%%%%%%%%%%%%%%%%%%%%%%%%%
\geometry{a4paper}

%%%%%%%%%%%%%%%%%%%%%%%%%% Define some useful colors %%%%%%%%%%%%%%%%%%%%%%%%%%
\definecolor{ocre}{RGB}{243,102,25}
\definecolor{mygray}{RGB}{243,243,244}
\definecolor{deepGreen}{RGB}{26,111,0}
\definecolor{shallowGreen}{RGB}{235,255,255}
\definecolor{deepBlue}{RGB}{61,124,222}
\definecolor{shallowBlue}{RGB}{235,249,255}
%%%%%%%%%%%%%%%%%%%%%%%%%%%%%%%%%%%%%%%%%%%%%%%%%%%%%%%%%%%%%%%%%%%%%%%%%%%%%%%

%%%%%%%%%%%%%%%%%%%%%%%%%% Define an orangebox command %%%%%%%%%%%%%%%%%%%%%%%%
\newcommand\orangebox[1]{\fcolorbox{ocre}{mygray}{\hspace{1em}#1\hspace{1em}}}
%%%%%%%%%%%%%%%%%%%%%%%%%%%%%%%%%%%%%%%%%%%%%%%%%%%%%%%%%%%%%%%%%%%%%%%%%%%%%%%

%%%%%%%%%%%%%%%%%%%%%%%%%%%% English Environments %%%%%%%%%%%%%%%%%%%%%%%%%%%%%
\newtheoremstyle{mytheoremstyle}{3pt}{3pt}{\normalfont}{0cm}{\rmfamily\bfseries}{}{1em}{{\color{black}\thmname{#1}~\thmnumber{#2}}\thmnote{\,--\,#3}}
\newtheoremstyle{myproblemstyle}{3pt}{3pt}{\normalfont}{0cm}{\rmfamily\bfseries}{}{1em}{{\color{black}\thmname{#1}~\thmnumber{#2}}\thmnote{\,--\,#3}}
\theoremstyle{mytheoremstyle}
\newmdtheoremenv[linewidth=1pt,backgroundcolor=shallowGreen,linecolor=deepGreen,leftmargin=0pt,innerleftmargin=20pt,innerrightmargin=20pt,]{theorem}{Theorem}[section]
\theoremstyle{mytheoremstyle}
\newmdtheoremenv[linewidth=1pt,backgroundcolor=shallowBlue,linecolor=deepBlue,leftmargin=0pt,innerleftmargin=20pt,innerrightmargin=20pt,]{definition}{Definition}[section]
\theoremstyle{myproblemstyle}
\newmdtheoremenv[linecolor=black,leftmargin=0pt,innerleftmargin=10pt,innerrightmargin=10pt,]{problem}{Problem}[section]
%%%%%%%%%%%%%%%%%%%%%%%%%%%%%%%%%%%%%%%%%%%%%%%%%%%%%%%%%%%%%%%%%%%%%%%%%%%%%%%

%%%%%%%%%%%%%%%%%%%%%%%%%%%%%%% Plotting Settings %%%%%%%%%%%%%%%%%%%%%%%%%%%%%
\usepgfplotslibrary{colorbrewer}
\pgfplotsset{width=8cm,compat=1.9}
%%%%%%%%%%%%%%%%%%%%%%%%%%%%%%%%%%%%%%%%%%%%%%%%%%%%%%%%%%%%%%%%%%%%%%%%%%%%%%%

%%%%%%%%%%%%%%%%%%%%%%%%%%%%%%% Title & Author %%%%%%%%%%%%%%%%%%%%%%%%%%%%%%%%
\title{Homework 2}
\author{Nguyen Tuan Anh - 2252038 - CN01}
%%%%%%%%%%%%%%%%%%%%%%%%%%%%%%%%%%%%%%%%%%%%%%%%%%%%%%%%%%%%%%%%%%%%%%%%%%%%%%%

\begin{document}
    \maketitle
    \section*{Section 2.1}
    \subsection*{Exercise 11}
        Determine whether each of these statements is true or false.
        \begin{enumerate} [label = (\alph*)]
            \item 0 $ \in \emptyset$
            \item $ \emptyset \in \{0\}$
            \item $ \{0\} \subset \emptyset $
            \item $ \emptyset \subset \{0\} $
            \item $ \{0\} \in \{0\} $
            \item $ \{0\} \subset \{0\} $
            \item $ \{\emptyset\} \subseteq \{\emptyset\}$
        \end{enumerate}
    \subsubsection*{Solution}
        \begin{enumerate} [label = (\alph*)]
            \item 0 $ \in \emptyset$: This statement is false because $\emptyset$
            has no elements so 0 can not be an element of the empty set.
            \item $ \emptyset \in \{0\}$: This statement is false because
            $ \emptyset $ is not an element in set \{0\}.
            \item $ \{0\} \subset \emptyset $: This statement is false because
            $ \emptyset $ has no elements so that \{0\} can not be a subset
            of $ \emptyset $.
            \item $ \emptyset \subset \{0\} $: This statement is true because
            $ \emptyset $ is one of the two sets that every nonempty set is
            guaranteed to have.
            \item $ \{0\} \in \{0\} $: This statement is false because 
            $ \{0\} $ is an element of \{\{0\}\} not an element of \{0\}.
            \item $ \{0\} \subset \{0\} $: This statement is false because the
            two set all have the same elements 0 so that is must be $ \subseteq $.
            \item $ \{\emptyset\} \subseteq \{\emptyset\}$: This statement is true
            because both singleton set have the same element $ \emptyset $. Therefore, this
            statement is true.
        \end{enumerate}
    \subsection*{Exercise 12}
        Determine whether these statements are true or false.
        \begin{enumerate} [label = (\alph*)]
            \item $ \emptyset \in \{\emptyset\} $
            \item $ \emptyset \in \{\emptyset,\{\emptyset\}\} $
            \item $ \{\emptyset\} \in \{\emptyset\} $
            \item $ \{\emptyset\} \in \{\{\emptyset\}\} $
            \item $ \{\emptyset\} \subset \{\emptyset, \{\emptyset\}\} $
            \item $ \{\{\emptyset\}\} \subset \{\emptyset, \{\emptyset\}\} $
            \item $ \{\{\emptyset\}\} \subset \{\{\emptyset\}, \{\emptyset\}\} $
        \end{enumerate}
    \subsubsection*{Solution}
        \begin{enumerate} [label = (\alph*)]
            \item \(\emptyset \in \{\emptyset\}\): This statement is true
            because \(\emptyset\) is an element of a singleton set contains 
            element \(\emptyset\).
            \item \(\emptyset \in \{\emptyset,\{\emptyset\}\}\): This statement
            is true because \(\emptyset\) is an element of the set 
            $\{\emptyset,\{\emptyset\}\}$.
            \item $ \{\emptyset\} \in \{\emptyset\} $: This statement is false
            because \(\emptyset\) must be an element of \(\{\{\emptyset\}\}\).
            \item $ \{\emptyset\} \in \{\{\emptyset\}\} $: This statement is true
            because the set \(\{\{\emptyset\}\}\) contains \(\{\emptyset\}\).
            \item $ \{\emptyset\} \subset \{\emptyset, \{\emptyset\}\} $: This
            statement is true because \(\emptyset\) is an element is the set \(\{\emptyset, \{\emptyset\}\}\)
            so the set contains \(\emptyset\) is a subset of $\{\emptyset, \{\emptyset\}\}$.
            \item $ \{\{\emptyset\}\} \subset \{\emptyset, \{\emptyset\}\} $: This
            statement is true and its reason is the same as problem (e).
            \item $ \{\{\emptyset\}\} \subset \{\{\emptyset\}, \{\emptyset\}\} $:
            We can see that the set $\{\{\emptyset\}, \{\emptyset\}\}$ has two elements which
            are equal to each other. Therefore, we can simplify it to \(\{\{\emptyset\}\}\).
            Therefore, this statement is false because these sets are equal to each other so
            it must be \(\subseteq\) instead of \(\subset\).
        \end{enumerate}
    \subsection*{Exercise 13}
        Determine whether each of these statements is true or false.
        \begin{enumerate} [label = (\alph*)]
            \item \(x \in \{x\}\)
            \item \(\{x\} \subseteq \{x\}\)
            \item \(\{x\} \in \{x\}\)
            \item \(\{x\} \in \{\{x\}\}\)
            \item \(\emptyset \subseteq \{x\}\)
            \item \(\emptyset \in \{x\}\)
        \end{enumerate}
    \subsubsection*{Solution}
        \begin{enumerate} [label = (\alph*)]
            \item \(x \in \{x\}\): This statement is true because \(x\) is an element in set
            \(x\).
            \item \(\{x\} \subseteq \{x\}\): This statement is true.
            \item \(\{x\} \in \{x\}\): This statement is false because \(x\) is an element of 
            \(\{\{x\}\}\) not \(\{x\}\).
            \item \(\{x\} \in \{\{x\}\}\): This statement is true due to the reason from problem(c).
            \item \(\emptyset \subseteq \{x\}\): This statement is true according to \textbf{Theorem 1}.
            \item \(\emptyset \in \{x\}\): This statement is false because \(\emptyset\) is not 
            an element of set \(\{x\}\).
        \end{enumerate}
    \subsection*{Exercise 26}
        Determine whether each of these sets is the power set of 
        a set, where a and b are distinct elements.
        \begin{enumerate} 
            \item \(\emptyset\)
            \item \(\{\emptyset, \{a\}\}\)
            \item \(\{\emptyset, \{a\}, \{\emptyset, a\}\}\)
            \item \(\{\emptyset, \{a\}, \{b\}, \{a, b\}\}\)
        \end{enumerate}
    \subsubsection*{Solution}
\end{document}