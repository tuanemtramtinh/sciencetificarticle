%%%%%%%%%%%%%%%%%%%%%%%%%%%%% Define Article %%%%%%%%%%%%%%%%%%%%%%%%%%%%%%%%%%
\documentclass{article}
%%%%%%%%%%%%%%%%%%%%%%%%%%%%%%%%%%%%%%%%%%%%%%%%%%%%%%%%%%%%%%%%%%%%%%%%%%%%%%%

%%%%%%%%%%%%%%%%%%%%%%%%%%%%% Using Packages %%%%%%%%%%%%%%%%%%%%%%%%%%%%%%%%%%
\usepackage{geometry}
\usepackage{graphicx}
\usepackage{amssymb}
\usepackage{amsmath}
\usepackage{amsthm}
\usepackage{empheq}
\usepackage{mdframed}
\usepackage{booktabs}
\usepackage{lipsum}
\usepackage{graphicx}
\usepackage{color}
\usepackage{psfrag}
\usepackage{pgfplots}
\usepackage{bm}
\usepackage{bookmark}
\usepackage{indentfirst}
\usepackage{enumitem}
%%%%%%%%%%%%%%%%%%%%%%%%%%%%%%%%%%%%%%%%%%%%%%%%%%%%%%%%%%%%%%%%%%%%%%%%%%%%%%%

% Other Settings

%%%%%%%%%%%%%%%%%%%%%%%%%% Page Setting %%%%%%%%%%%%%%%%%%%%%%%%%%%%%%%%%%%%%%%
\geometry{a4paper}

%%%%%%%%%%%%%%%%%%%%%%%%%% Define some useful colors %%%%%%%%%%%%%%%%%%%%%%%%%%
\definecolor{ocre}{RGB}{243,102,25}
\definecolor{mygray}{RGB}{243,243,244}
\definecolor{deepGreen}{RGB}{26,111,0}
\definecolor{shallowGreen}{RGB}{235,255,255}
\definecolor{deepBlue}{RGB}{61,124,222}
\definecolor{shallowBlue}{RGB}{235,249,255}
%%%%%%%%%%%%%%%%%%%%%%%%%%%%%%%%%%%%%%%%%%%%%%%%%%%%%%%%%%%%%%%%%%%%%%%%%%%%%%%

%%%%%%%%%%%%%%%%%%%%%%%%%% Define an orangebox command %%%%%%%%%%%%%%%%%%%%%%%%
\newcommand\orangebox[1]{\fcolorbox{ocre}{mygray}{\hspace{1em}#1\hspace{1em}}}
%%%%%%%%%%%%%%%%%%%%%%%%%%%%%%%%%%%%%%%%%%%%%%%%%%%%%%%%%%%%%%%%%%%%%%%%%%%%%%%

%%%%%%%%%%%%%%%%%%%%%%%%%%%% English Environments %%%%%%%%%%%%%%%%%%%%%%%%%%%%%
\newtheoremstyle{mytheoremstyle}{3pt}{3pt}{\normalfont}{0cm}{\rmfamily\bfseries}{}{1em}{{\color{black}\thmname{#1}~\thmnumber{#2}}\thmnote{\,--\,#3}}
\newtheoremstyle{myproblemstyle}{3pt}{3pt}{\normalfont}{0cm}{\rmfamily\bfseries}{}{1em}{{\color{black}\thmname{#1}~\thmnumber{#2}}\thmnote{\,--\,#3}}
\theoremstyle{mytheoremstyle}
\newmdtheoremenv[linewidth=1pt,backgroundcolor=shallowGreen,linecolor=deepGreen,leftmargin=0pt,innerleftmargin=20pt,innerrightmargin=20pt,]{theorem}{Theorem}[section]
\theoremstyle{mytheoremstyle}
\newmdtheoremenv[linewidth=1pt,backgroundcolor=shallowBlue,linecolor=deepBlue,leftmargin=0pt,innerleftmargin=20pt,innerrightmargin=20pt,]{definition}{Definition}[section]
\theoremstyle{myproblemstyle}
\newmdtheoremenv[linecolor=black,leftmargin=0pt,innerleftmargin=10pt,innerrightmargin=10pt,]{problem}{Problem}[section]
%%%%%%%%%%%%%%%%%%%%%%%%%%%%%%%%%%%%%%%%%%%%%%%%%%%%%%%%%%%%%%%%%%%%%%%%%%%%%%%

%%%%%%%%%%%%%%%%%%%%%%%%%%%%%%% Plotting Settings %%%%%%%%%%%%%%%%%%%%%%%%%%%%%
\usepgfplotslibrary{colorbrewer}
\pgfplotsset{width=8cm,compat=1.9}
%%%%%%%%%%%%%%%%%%%%%%%%%%%%%%%%%%%%%%%%%%%%%%%%%%%%%%%%%%%%%%%%%%%%%%%%%%%%%%%

%%%%%%%%%%%%%%%%%%%%%%%%%%%%%%% Title & Author %%%%%%%%%%%%%%%%%%%%%%%%%%%%%%%%
\title{Homework 2}
\author{Nguyen Tuan Anh - 2252038 - CN01}
%%%%%%%%%%%%%%%%%%%%%%%%%%%%%%%%%%%%%%%%%%%%%%%%%%%%%%%%%%%%%%%%%%%%%%%%%%%%%%%

\begin{document}
    \maketitle
    \section*{Section 2.1}
    \subsection*{Exercise 11}
        Determine whether each of these statements is true or false.
        \begin{enumerate} [label = (\alph*)]
            \item 0 $ \in \emptyset$
            \item $ \emptyset \in \{0\}$
            \item $ \{0\} \subset \emptyset $
            \item $ \emptyset \subset \{0\} $
            \item $ \{0\} \in \{0\} $
            \item $ \{0\} \subset \{0\} $
            \item $ \{\emptyset\} \subseteq \{\emptyset\}$
        \end{enumerate}
    \subsubsection*{Solution}
        \begin{enumerate} [label = (\alph*)]
            \item 0 $ \in \emptyset$: This statement is false because $\emptyset$
            has no elements so 0 can not be an element of the empty set.
            \item $ \emptyset \in \{0\}$: This statement is false because
            $ \emptyset $ is not an element in set \{0\}.
            \item $ \{0\} \subset \emptyset $: This statement is false because
            $ \emptyset $ has no elements so that \{0\} can not be a subset
            of $ \emptyset $.
            \item $ \emptyset \subset \{0\} $: This statement is true because
            $ \emptyset $ is one of the two sets that every nonempty set is
            guaranteed to have.
            \item $ \{0\} \in \{0\} $: This statement is false because 
            $ \{0\} $ is an element of \{\{0\}\} not an element of \{0\}.
            \item $ \{0\} \subset \{0\} $: This statement is false because the
            two set all have the same elements 0 so that is must be $ \subseteq $.
            \item $ \{\emptyset\} \subseteq \{\emptyset\}$: This statement is true
            because both singleton set have the same element $ \emptyset $. Therefore, this
            statement is true.
        \end{enumerate}
    \subsection*{Exercise 12}
        Determine whether these statements are true or false.
        \begin{enumerate} [label = (\alph*)]
            \item $ \emptyset \in \{\emptyset\} $
            \item $ \emptyset \in \{\emptyset,\{\emptyset\}\} $
            \item $ \{\emptyset\} \in \{\emptyset\} $
            \item $ \{\emptyset\} \in \{\{\emptyset\}\} $
            \item $ \{\emptyset\} \subset \{\emptyset, \{\emptyset\}\} $
            \item $ \{\{\emptyset\}\} \subset \{\emptyset, \{\emptyset\}\} $
            \item $ \{\{\emptyset\}\} \subset \{\{\emptyset\}, \{\emptyset\}\} $
        \end{enumerate}
    \subsubsection*{Solution}
        \begin{enumerate} [label = (\alph*)]
            \item \(\emptyset \in \{\emptyset\}\): This statement is true
            because \(\emptyset\) is an element of a singleton set contains 
            element \(\emptyset\).
            \item \(\emptyset \in \{\emptyset,\{\emptyset\}\}\): This statement
            is true because \(\emptyset\) is an element of the set 
            $\{\emptyset,\{\emptyset\}\}$.
            \item $ \{\emptyset\} \in \{\emptyset\} $: This statement is false
            because \(\emptyset\) must be an element of \(\{\{\emptyset\}\}\).
            \item $ \{\emptyset\} \in \{\{\emptyset\}\} $: This statement is true
            because the set \(\{\{\emptyset\}\}\) contains \(\{\emptyset\}\).
            \item $ \{\emptyset\} \subset \{\emptyset, \{\emptyset\}\} $: This
            statement is true because \(\emptyset\) is an element is the set \(\{\emptyset, \{\emptyset\}\}\)
            so the set contains \(\emptyset\) is a subset of $\{\emptyset, \{\emptyset\}\}$.
            \item $ \{\{\emptyset\}\} \subset \{\emptyset, \{\emptyset\}\} $: This
            statement is true and its reason is the same as problem (e).
            \item $ \{\{\emptyset\}\} \subset \{\{\emptyset\}, \{\emptyset\}\} $:
            We can see that the set $\{\{\emptyset\}, \{\emptyset\}\}$ has two elements which
            are equal to each other. Therefore, we can simplify it to \(\{\{\emptyset\}\}\).
            Therefore, this statement is false because these sets are equal to each other so
            it must be \(\subseteq\) instead of \(\subset\).
        \end{enumerate}
    \subsection*{Exercise 13}
        Determine whether each of these statements is true or false.
        \begin{enumerate} [label = (\alph*)]
            \item \(x \in \{x\}\)
            \item \(\{x\} \subseteq \{x\}\)
            \item \(\{x\} \in \{x\}\)
            \item \(\{x\} \in \{\{x\}\}\)
            \item \(\emptyset \subseteq \{x\}\)
            \item \(\emptyset \in \{x\}\)
        \end{enumerate}
    \subsubsection*{Solution}
        \begin{enumerate} [label = (\alph*)]
            \item \(x \in \{x\}\): This statement is true because \(x\) is an element in set
            \(x\).
            \item \(\{x\} \subseteq \{x\}\): This statement is true.
            \item \(\{x\} \in \{x\}\): This statement is false because \(x\) is an element of 
            \(\{\{x\}\}\) not \(\{x\}\).
            \item \(\{x\} \in \{\{x\}\}\): This statement is true due to the reason from problem(c).
            \item \(\emptyset \subseteq \{x\}\): This statement is true according to \textbf{Theorem 1}.
            \item \(\emptyset \in \{x\}\): This statement is false because \(\emptyset\) is not 
            an element of set \(\{x\}\).
        \end{enumerate}
    \subsection*{Exercise 26}
        Determine whether each of these sets is the power set of 
        a set, where a and b are distinct elements.
        \begin{enumerate} [label = (\alph*)]
            \item \(\emptyset\)
            \item \(\{\emptyset, \{a\}\}\)
            \item \(\{\emptyset, \{a\}, \{\emptyset, a\}\}\)
            \item \(\{\emptyset, \{a\}, \{b\}, \{a, b\}\}\)
        \end{enumerate}
    \subsubsection*{Solution}
        The set(d) is the power set of set \{a, b\} because the set has two elements a and b
        so that its power set has \(2^2 = 4\) elements, which has the same number of elements
        of set(d). 
        \begin{align*}
            \mathcal{P}(\{a, b\}) = \{\emptyset, \{a\}, \{b\}, \{a, b\}\}
        \end{align*}
    \subsection*{Exercise 27}
        Prove that \(\mathcal{P}(A) \subseteq \mathcal{P}(B)\) if and only if \(A\subseteq B\).
    \subsubsection*{Solution}
        There are two things we need to prove:
        \begin{align*}
            (\mathcal{P}(A) \subseteq \mathcal{P}(B) \to A \subseteq B) \land (A \subseteq B \to \mathcal{P}(A) \subseteq \mathcal{P}(B))
        \end{align*}
        \begin{itemize}
            \item \(\mathcal{P}(A) \subseteq \mathcal{P}(B) \to A \subseteq B\):\\
            
            \(\mathcal{P}(A) \subseteq \mathcal{P}(B)\) means that every element of power set A
            is also an element of power set B. Addtionally, we all know that power set of a set
            has \(2^n\) elements created from the combinations of all the elements from the original
            set. Because all every element of power set A is also an element of power set B so we can infer that
            the element of set A is also an element of set B because they have the same combinations in the power set.
            Therefore, this case is true.
            \item \(A \subseteq B \to \mathcal{P}(A) \subseteq \mathcal{P}(B)\)\\
            
            \(A \subseteq B\) means that every element of A is also an element of B. Because A and B
            have the same element so that there combinations of elements of these two sets will be the same.
            Therefore, the elements power set of A and B will be the same because both of them contain
            all subsets of A and B(We have that \(A \subseteq B\)). Therefore, this case is true.
        \end{itemize}
        Because both cases are true so that \((\mathcal{P}(A) \subseteq \mathcal{P}(B) \to A \subseteq B) \land (A \subseteq B \to \mathcal{P}(A) \subseteq \mathcal{P}(B))\)
        is true and it is equivalent to \(\mathcal{P}(A) \subseteq \mathcal{P}(B) \leftrightarrow A \subseteq B\).
    \subsection*{Exercise 28} Show that if \(A \subseteq C\) and \(B \subseteq D\), then \(A \times B \subseteq C \times D\)
    \subsubsection*{Solution} Because A is a subset of C and B is a subset of D. Suppose that we have set A, B, C and D:
        \begin{itemize}
            \item \(A = \{a, b\}\)
            \item \(B = \{c, d\}\)
            \item \(C = \{a, b\}\)
            \item \(D = \{c, d\}\)
        \end{itemize}
        We have that: 
        \begin{align*}
            A \times B = \{(a,c),(a,d),(b,c),(b,d)\}\\
            C \times D = \{(a,c),(a,d),(b,c),(b,d)\}
        \end{align*}
        After using \textbf{Cartesian Product} to calculate \(A \times B\), \(C \times D\), we can
        see that every element \(A \times B\) is also the element of \(C \times D\). Therefore,
        we can infer that \(A \times B \subseteq C \times D\).
    \subsection*{Exercise 41}
        Explain why $A \times B \times C$ and $(A \times B) \times C$ are not the same.
    \subsubsection*{Solution}
        Let \(A = \{0, 1\}, B = \{1, 2\}, C = \{0, 1, 2\}\)
        \begin{align*}
            A \times B \times C &= \{(0, 1, 0), (0, 1, 1), (0, 1, 2), (0, 2, 0), (0, 2, 1),\\
            &(0, 2, 2),(1, 1, 0), (1, 1, 1), (1, 1, 2), (1, 2, 0), (1, 2, 1), (1, 2, 2)\}\\
            A \times B &= \{(0,1), (0,2), (1,1), (1,2)\}\\
            (A \times B) \times C &= \{((0,1),0,),((0, 1), 1), ((0, 1), 2), ((0, 2), 0), ((0, 2), 1),\\
            &((0, 2), 2), ((1, 1), 0), ((1, 1), 1), ((1, 1), 2), ((1, 2), 0), ((1, 2), 1), ((1, 2), 2)\}
        \end{align*}
        As we can see that \(A \times B \times C\) gives us  a set of 3-tuples has a form (a, b, c) with 
        \(a \in A, b \in B, c \in C\). However, \((A \times B) \times C\) gives us a set of 2-tuples has
        a form ((a, b), c) with \((a,b) \in A \times B\) and \(c \in C\). This is different from \(A \times B \times C\)
        because \(A \times B \times C\) is a 3-tuples but \((A \times B) \times C\) is a 2-tuples which has the
        first element is a ordered pair of \(A \times B\). Therefore, \(A \times B \times C\) is different from \((A \times B) \times C\)
    \subsection*{Exercise 42}
        Explain why\((A\times B)\times(C\times D)\)and \(A\times(B\times C)\times D\) are not the same.
    \subsubsection*{Solution}
        Let a, b, c, d are elements of set A, B, C, D respectively. Therefore, we have that
        \(a \in A, b \in B, c \in C, d \in D\). We get that \(A \times B\) will be a set
        consists ordered pairs (a, b) and \(C \times D\) also consists ordered pair (c, d).
        Because (a,b) and (c,d) is an element of set \(A \times B\) and \(C \times D\).
        Therefore, if we do the Cartesian product \((A\times B)\times (C\times D)\). The result
        will be a new set consists ordered pair ((a,b),(c,d)) which has the two elements are
        two ordered pairs from \(A \times B\) and \(C \times D\).\\
        
        However, \(A \times (B\times C) \times D\) is completely different. We get \(B \times C\) will be a set consists ordered
        pair (b, c) and continue to calculate \(A \times (B \times C) \times D\), we will get a new set consists 3-tuples
        (a,(b,c),d) with the fist element is an element \(\in A\), the second element is the element \(\in (B \times C)\), and
        the last element is an element \(\in D\).\\

        Therefore, we can infer that \((A \times B)\times(C \times D)\) and \(A \times (B \times C)\times D\) are different.
    \subsection*{Exercise 43}
        Prove or disprove that if A and B are sets, then \(\mathcal{P}(A \times B) = \mathcal{P}(A) \times \mathcal{P}(B)\)
    \subsubsection*{Solution}
        Let \(A = \{0, 1\}\) and \(B = {1, 2}\). Therefore, we get that \(|A| = |B| = 2\) and \(|A \times B| = |A| \times |B| = 2 \times 2 = 4\).
        Because \(A \times B\) has 4 elements so that \(\mathcal{P}(A)\) will have \(2^{|A|\times|B|} = 2^4 = 16\). However, we get that
        \(|\mathcal{P}(A)| = 2^{|A|} = 2^2 = 4\) and \(|\mathcal{P}(B)| = 2^{|B|} = 2^2 = 4\) and
        \begin{align*}
            |\mathcal{P}(A) \times \mathcal{P}(B)| = 2^{|A|}\times 2^{|B|} = 2^{|A| + |B|} = 2^4 =16
        \end{align*}
        Although the result of \(|\mathcal{P}(A\times B)|\) and \(|\mathcal{P}(A) \times \mathcal{P}(B)|\) are both equal to
        16 but the way they give the result 16 are completely different.
        \begin{align*}
            |\mathcal{P}(A\times B)| = 2^{|A| \times |B|} = 2^4 = 16\\
            |\mathcal{P}(A) \times \mathcal{P}(B)| = 2^{|A| + |B|} = 16
        \end{align*}
        Because \(2^{|A| \times |B|} \neq 2^{|A| + |B|}\) so we can conclude that \(\mathcal{P}(A \times B) \neq \mathcal{P}(A) \times \mathcal{P}(B)\)
    \subsection*{Exercise 44}
        Prove or disprove that if A, B, and C are nonempty sets and \(A \times B = A \times C\), then \(B = C\).
    \subsubsection*{Solution}
        We know that \(A \times B\) and \(A \times C\) will create a set whose element is ordered pairs. 
        Let \(a \in A, b \in B, c \in C\) so that \(A \times B\) will be \((a, b)\) and \(A \times C\) will be \((a, c)\).
        Because we have that \(A \times B = A \times C\) so that \((a,b) = (a,c)\). \((a,b) = (a,c)\) if and only if a = a and b = c.
        Addtionally, \(b \in B, c \in C\) so that \(B = C\).
    \subsection*{Exercise 49}
    The defining property of an ordered pair is that two ordered pairs are equal if and only if their first elements are
    equal and their second elements are equal. Surprisingly, instead of taking the ordered pair as a primitive concept, we can construct ordered pairs using basic notions from set theory. Show that if we define the ordered pair
    (a, b) to be \{\{a\}, \{a, b\}\}, then (a, b) = (c, d) if and only if a = c and b = d.
    \subsubsection*{Solution}
        Define the ordered pairs (a,b) and (c,d) to be\{\{a\},\{a,b\}\} and \{\{c\},\{c,d\}\}. We need to prove that
        \(\{\{a\},\{a,b\}\} = \{\{c\},\{c,d\}\} \leftrightarrow a = b \ and \ c = d\)\\.
        
        Firstly, we prove that \(\{\{a\},\{a,b\}\} = \{\{c\},\{c,d\}\} \to a = b \ and \ c = d\), suppose that \(\{\{a\},\{a,b\}\} = \{\{c\},\{c,d\}\}\) so that \({a} = {c}\) and we can infer that \(a = c\). For \(\{a,b\} = \{c,d\}\), we have that
        \(a = c\) so that \(b = d\).\\

        Secondly, we prove that \(a = b \ and \ c = d \to \{\{a\},\{a,b\}\} = \{\{c\},\{c,d\}\}\), suppose that \(a = c\) and \(b = d\) so that we can infer \(\{a\} \subseteq \{c\}\) and \(\{c\} \subseteq \{a\}\) so that
        \(\{a\} = \{c\}\). Because \(a = c\) and \(b = d\) it result in \(\{a,b\} \subseteq \{c,d\}\) and \(\{c,d\} \subseteq \{a,b\}\), which means that
        \(\{a,b\} = \{c,d\}\). We have \(\{a\} = \{c\}\) and \(\{a,b\} = \{c,d\}\), we infer that \(\{\{a\},\{a,b\}\} \subseteq \{\{c\},\{c,d\}\}\)
        and \(\{\{c\},\{c,d\}\} \subseteq \{\{a\},\{a,b\}\}\). Therefore, \(\{\{a\},\{a,b\}\} = \{\{c\},\{c,d\}\}\).\\

        Because two cases above are both true so we can conclude that \(\{\{a\},\{a,b\}\} = \{\{c\},\{c,d\}\} \leftrightarrow a = b \ and \ c = d\). Moreover, because the ordered
        pair is defined in \{\{a\},\{a, b\}\} so that it can be transfered to \((a,b) = (c,d) \leftrightarrow a = b \ and \ c = d\).
    \subsubsection*{Exercise 50}
        This exercise presents Russell's paradox. Let S be the that contains a set \(x\) if the set \(x\) does not belong to itself, so that \(S = \{x | x \notin x\}\).
        \begin{enumerate} [label = (\alph*)]
            \item Show the assumption that S is a member of S leads to a contradiction.
            \item Show the assumption that S is not a member of S leads to a contradiction.
        \end{enumerate}
    \subsubsection*{Solution}
        \begin{enumerate} [label = (\alph*)]
            \item If \(S\) is a member of \(S\) it means that \(S\) belongs to itself(\(S \in S\)). However, we define that \textbf{S be the that contains a set x if the set x does
            not belong to itself} so that we get \(S \notin S\) but the assumption is \(S \in S\), therefore, it leads to a contradiction.
            \item If \(S\) not a member of \(S\), it satisfies with the hypothesis \textbf{S be the that contains a set x if the set x does
            not belong to itself} so that we get \(S \in S\). However, the assumption is \(S \notin S\) so it leads to a contradiction.
        \end{enumerate}
    \subsection*{Exercise 51}
        Describe a procedure for listing all the subsets of a finite set.
    \subsubsection*{Solution}
        To list all the subsets of a finite set, we will use the binary to represent it. Let \(X\) is a set contains
        \(\{a_1, a_2, a_3, ..., a_n\}\). We will represent these elements of the set in binary, with n elements, it will be 
        n digits of the binary and the number of subset will be \(2^n - 1\). The order of the digits in binary is the same as
        the order of the elements in the subset. If digits \textit{i(i\(\leqslant\) n)} is 1 so it in a subset and if it is 0, it will be remove from the subset.\\
        
        For example, \(X = \{1, 2, 3\}\). Because there are 3 elements in the set \textit{X} so that it will be \(2^3 - 1 = 7\) subsets and 3 binary digits.
        \begin{align*}
            000 &= \{\}\\
            001 &= \{3\}\\
            010 &= \{2\}\\
            011 &= \{2, 3\}\\
            100 &= \{1\}\\
            101 &= \{1, 3\}\\
            110 &= \{1, 2\}\\
            111 &= \{1, 2, 3\}
        \end{align*}

    \section*{Section 2.2}
    \subsection*{Exercise 15}
        Prove the second De Morgan law in Table 1 by showing that if A and B are sets, then \(\overline{A \cap B} = \overline{A} \cup \overline{B}\)
        \begin{enumerate} [label = (\alph*)]
            \item by showing each side is a subset of the other side.
            \item using membership table.
        \end{enumerate}
    \subsubsection*{Solution}
        \begin{enumerate} [label = (\alph*)]
            \item To show each side is a subset of other side, we will need to prove that \(\overline{A \cap B} \subseteq \overline{A} \cup \overline{B}\)
            and \(\overline{A} \cup \overline{B} \subseteq \overline{A \cap B}\)
            \begin{itemize}
                \item \(\overline{A \cap B} \subseteq \overline{A} \cup \overline{B}\): Suppose x is an element
                in \(\overline{A \cap B}\) so \(x \in \overline{A \cap B}\). Therefore, \(x \notin A \cap B\) and we can transfer
                it into \(\lnot((x \in A) \land (x \in B))\). Applying De Morgan law, it will be \(\lnot(x \in A) \lor \lnot(x \in B)\). 
                Using the definition of negation of propositions, we have \(x \notin A\) or \(x \notin B\) and can be transfered to
                \(x \in \overline{A}\) or \(x \in \overline{B}\). From this, we can infer that \(x \in \overline{A} \cup \overline{B}\).
                Because \(x \in \overline{A \cap B}\) and also \(x \in \overline{A} \cup \overline{B}\) so that \(\overline{A \cap B} \subseteq \overline{A} \cup \overline{B}\).
                \item \(\overline{A} \cup \overline{B} \subseteq \overline{A \cap B}\): Suppose x is an element 
                in \(\overline{A} \cup \overline{B}\) so \(x \in \overline{A} \cup \overline{B}\). Therefore, \(x \notin A\) or \(x \notin B\)
                and we can transfer it to \(\lnot(x\in A) \lor \lnot(x\in B)\). Applying De Morgan law, it will be \(\lnot((x \in A) \land (x \in B))\).
                By the definition of intersection, it follows that \(\lnot (x \in A \cap B)\), which means that \(x \notin A \cap B\) and
                we can infer that \(x \in \overline{A \cap B}\). Because \(x \in \overline{A} \cup \overline{B}\) and
                also \(x \in \overline{A \cap B}\) so that \(\overline{A} \cup \overline{B} \subseteq \overline{A \cap B}\).
            \end{itemize}
            Because we have shown that each set is a subset of the other, so \(\overline{A \cap B} = \overline{A} \cup \overline{B}\).
            \item Using membership table:\\
                \begin{center}
                    \begin{tabular}{|c|c|c|c|c|c|c|}
                        \hline
                        A & B & \(\overline{A}\) & \(\overline{B}\) & \(A \cap B\) & \(\overline{A \cap B}\) & \(\overline{A} \cup \overline{B}\)\\
                        \hline
                        1 & 1 & 0 & 0 & 1 & 0 & 0\\
                        \hline
                        1 & 0 & 0 & 1 & 0 & 1 & 1\\
                        \hline
                        0 & 1 & 1 & 0 & 0 & 1 & 1\\
                        \hline
                        0 & 0 & 1 & 1 & 1 & 1 & 1\\
                        \hline
                    \end{tabular}
                \end{center}
            From the membership table, we see that the value of \(\overline{A \cap B}\) and \(\overline{A} \cup \overline{B}\) are the same so that we can
            conclude that \(\overline{A \cap B} = \overline{A} \cup \overline{B}\).
        \end{enumerate}
    \subsection*{Exercise 16}
        Let A and B be sets. Show that
        \begin{enumerate} [label = (\alph*)]
            \item \((A \cap B) \subseteq A\)
            \item \(A \subseteq (A \cup B)\)
            \item \(A - B \subseteq A\)
            \item \(A \cap (B - A) = \emptyset\)
            \item \(A \cup (B - A) = A \cup B\)
        \end{enumerate}
    \subsubsection*{Solution}
        \begin{enumerate} [label = (\alph*)]
            \item We transfer \(A \cap B\) in propositions, it will be \((x \in A) \land (x \in B)\).
            Because in \(A \cap B\), we have \(x \in A \) so that we can infer that \(A \cap B\) is a subset of \(A\).
            \item We transfer \(A \cup B\) in propositions, it will be \((x \in A) \lor (x \in B)\).
            Because \(x\) is an element of \(A\) and in \(A \cup B\), \(x\) also an element of \(A\) so \(A\) is a subset of \(B\).
            \item We transfer \(A - B = A \cap \overline{B}\) in propositions, it will be \((x \in A) \land (x \notin B)\).
            In \(A \cap \overline{B}\) we have \(x \in A\) so that \(x\) is an element in \(A \cap \overline{B}\) and also an element of A.
            So \(A \cap \overline{B}\) is a subset of \(A\).
            \item Because \(B - A = B \cap \overline{A}\). Transfer it into propositions, it will be \((x \in B) \land (x \notin A)\).
            \(x\) not an element of \(B - A\) so that \(x\) can not be an elemnent of \(A\). Therefore, if we
            use intersection operation, it will be \((x \in A) \land (x \in B) \land (x \notin A)\). We see that it is a
            contradiction because \(x\) can not be both an element of \(A\) and do not belong to \(A\). Thus, it will be an \(\emptyset\).
            \item Transfer \(B - A = B \cap \overline{A}\). Then, it will be \(A \cup (B \cap \overline{A}) = (A \cup B) \cap (A \cup 
            \overline{A}) = (A \cup B) \cap U = A \cup B\).
        \end{enumerate}
    \subsection*{Exercise 17}
        Show that if \(A\) and \(B\) are sets in a universe \(U\) then \(A \subseteq B\) if and only if \(\overline{A} \cup B = U\).
    \subsubsection*{Solution}
        \begin{itemize}
            \item \(A \subseteq B \to \overline{A}\cup B=U\): \(A \subseteq B\) means \(x\) is an element of \(A\) and \(x\) also an element of \(B\).
            We get that \(\overline{A} \cup B\) is \(x \notin A \lor x \in B\). Because \(x \in B\) and \(x\) also in A due to \(A \subseteq B\). Moreover, \(A\)
            and B are sets in universe \(U\) so that \(\overline{A} \cup B\) still in be in universe.
            \item \(\overline{A} \cup B = U \to A \subseteq B\): From the complement laws, we have that \(\overline{A} \cup A = U\). Therefore,
            if \(\overline{A} \cup B = U\) then \(x\) must be an elemnent of \(A\) and also an element of \(B\). Therefore, we have that
            \((x \in A) \land (x \in B)\), which means that \(A \subseteq B\).
        \end{itemize}
        Because both cases above are all make sense so we can conclude that if \(A\) and \(B\) are sets in a universe \(U\) then
        \begin{align*}
            A \subseteq B \leftrightarrow \overline{A} \cup B = U 
        \end{align*}
    \subsection*{Exercise 19}
        Show that if A, B, and C are sets, then \(\overline{A \cap B \cap C} = \overline{A} \cup \overline{B} \cup \overline{C}\)
        \begin{enumerate} [label = (\alph*)]
            \item by showing each side is a subset of other side.
            \item using a membership table.
        \end{enumerate}
    \subsubsection*{Solution}
        \begin{enumerate} [label = (\alph*)]
            \item We need to prove that \(\overline{A \cap B \cap C} \subseteq \overline{A} \cup \overline{B} \cup \overline{C}\) and conversely.\\
            
            Firstly, suppose \(x\) is an element of \(\overline{A \cap B \cap C}\), we transfer \(\overline{A \cap B \cap C}\) in propositions, it will be \(\lnot((x \in A) \land (x \in B) \land (x \in C))\). Applying the
            De Morgan law in propositions, we transfer it to \(\lnot(x \in A) \lor \lnot(x \in B) \lor \lnot(x \in C)\). Moreover, it is also equivalent
            to \((x \notin A) \lor (x \notin B) \lor (x \notin C) \equiv (x \in \overline{A}) \lor (x \in \overline{B}) \lor (x \in \overline{C})\). 
            According to the definition of \textit{union}, \((x \in \overline{A}) \lor (x \in \overline{B}) \lor (x \in \overline{C}) = \overline{A} \cup \overline{B} \cup \overline{C}\).
            Therefore, \(\overline{A \cap B \cap C} \subseteq \overline{A} \cup \overline{B} \cup \overline{C}\).\\

            Secondly, suppose \(x\) is an element of \(\overline{A} \cup \overline{B} \cup \overline{C}\)we transfer \(\overline{A} \cup \overline{B} \cup \overline{C}\) in propositions, it will be
            \((x \in \overline{A}) \lor (x \in \overline{B}) \lor (x \in \overline{C}) \equiv (x \notin A) \lor (x \notin B) \lor (x \notin C) \equiv \lnot(x \in A) \lor \lnot(x \in B) \lor \lnot(x \in C)\). 
            Applying De Morgan law in propositions, we transfer it to \(\lnot((x \in A) \land (x \in B) \land (x \in C))\). By 
            the definition of intersection, it will be \(\lnot(A \cap B \cap C)\), which means that \(x \notin A \cap B \cap C\).
            Therefore, we can infer that \(x \in \overline{A \cap B \cap C}\). Thus, \(\overline{A} \cup \overline{B} \cup \overline{C} \subseteq \overline{A \cap B \cap C}\).

            Because we have proved that \(\overline{A \cap B \cap C} \subseteq \overline{A} \cup \overline{B} \cup \overline{C}\) and \(\overline{A} \cup \overline{B} \cup \overline{C} \subseteq \overline{A \cap B \cap C}\) so that
            we can infer \(\overline{A \cap B \cap C} = \overline{A} \cup \overline{B} \cup \overline{C}\).
            \item We have the membership table:\\
                \begin{center}
                    \begin{tabular}{|c|c|c|c|c|c|c|c|}
                        \hline 
                        A & B & C & \(\overline{A}\) & \(\overline{B}\) & \(\overline{C}\) & \(\overline{A \cap B \cap C}\) & \(\overline{A} \cup \overline{B} \cup \overline{C}\)\\
                        \hline 
                        1 & 1 & 1 & 0 & 0 & 0 & 0 & 0\\
                        \hline
                        1 & 1 & 0 & 0 & 0 & 1 & 1 & 1\\
                        \hline 
                        1 & 0 & 1 & 0 & 1 & 0 & 1 & 1\\
                        \hline
                        1 & 0 & 0 & 0 & 1 & 1 & 1 & 1\\
                        \hline
                        0 & 1 & 1 & 1 & 0 & 0 & 1 & 1\\
                        \hline
                        0 & 1 & 0 & 1 & 0 & 1 & 1 & 1\\
                        \hline
                        0 & 0 & 1 & 1 & 0 & 0 & 1 & 1\\
                        \hline 
                        0 & 0 & 0 & 1 & 1 & 1 & 1 & 1\\
                        \hline
                    \end{tabular}
                \end{center}
        \end{enumerate}
    \subsection*{Exercise 20}
        Let A, B, and C be sets. Show that:
        \begin{enumerate} [label = (\alph*)]
            \item \((A \cup B) \subseteq (A \cup B \cup C)\)
            \item \((A \cap B \cap C) \subseteq (A \cap B)\)
            \item \((A - B) - C \subseteq A - C\)
            \item \((A - C) \cap (C - B) = \emptyset\) 
            \item \((B - A) \cup (C - A) = (B \cup C) - A\)
        \end{enumerate} 
    \subsubsection*{Solution}
        \begin{enumerate} [label = (\alph*)]
            \item Suppose that \(x \in (A \cup B)\), in propositions it is \((x \in A) \lor (x \in B)\) and \(A \cup B \cup C\) in
            propositions will be \((x \in A) \lor (x \in B) \lor (x \in C)\). Because in \(A \cup B \cup C\), we also have that
            \(x \in A\) and \(x \in B\). Therefore, \((A \cup B) \subseteq (A \cup B \cup C)\).
            \item Suppose that  \(x \in (A \cap B \cap C)\) so \(x\) is an element that A, B, C all have. In \(A \cap B\),
            \(x\) is the elements that both A and B have. Therefore, the element in \(A \cap B \cap C\) will also the element in
            \(A \cap B\). Thus, \((A \cap B \cap C) \subseteq (A \cap B)\).
            \item \((A - B) - C = (A \cap \overline{B}) \cap \overline{C} = A \cap \overline{B} \cap \overline{C}\). Suppose
            that \(x \in (A \cap \overline{B} \cap \overline{C})\) so \(x\) is an element that \(A, \overline{B}, \overline{C}\).
            In \(A - C = A \cap \overline{C}\), \(x\) is the element that both \(A, \overline{C}\) have. Therefore, the element in
            \((A - B) - C\) will also the element in \(A - C\). Thus, \((A - B) - C \subseteq A - C\).
            \item \((A - C) \cap (B - C) = A \cap \overline{C} \cap C \cap \overline{B}\). According to \textbf{Complement laws}, \(C \cap \overline{C} = \emptyset\).
            Due to \textbf{Identity laws}, \(A \cap \overline{B} \cap \emptyset = \emptyset\).
            \item We have that: 
            \begin{align*}
                (B - A) \cup (C - A) &= (B \cap \overline{A}) \cup (C \cap \overline{A})\\
                (B \cup C) - A &= (B \cup C) \cap \overline{A} \\
                                &= (B \cap \overline{A}) \cup (C \cap \overline{A})\\
            \end{align*}
            Because both of them are equal to \((B \cap \overline{A}) \cup (C \cap \overline{A})\) so that 
            \((B - A) \cup (C - A) = (B \cup C) - A\).
        \end{enumerate}
    \subsection*{Exercise 21}
        Show that if \(A\) and \(B\) are sets, then
        \begin{enumerate} [label = (\alph*)]
            \item \(A - B = A \cap \overline{B}\)
            \item \((A \cap B) \cup (A \cap \overline{B}) = A\)
        \end{enumerate}
    \subsubsection*{Solution}
        \begin{enumerate} [label = (\alph*)]
            \item \(A - B = A \cap \overline{B}\): We got that \(A - B\) is the difference of \(A\) and \(B\).
            According to the definition of \textit{difference}, transfer in propositions \((x \in A) \land (x \notin B)\), which is 
            the same as \(A \cap \overline{B}\) because in propositions, \(A \cap \overline{B}\) is \((x \in A) \cap (x \in \overline{B})\).
            \(x \in \overline{B}\) means \(x\) is the element not in \(B\). Therefore, \(A - B = A \cap \overline{B}\).
            \item \((A \cap B) \cup (A \cap \overline{B}) = A\) 
            \begin{align*}
                (A \cap B) \cup (A \cap \overline{B}) &= ((A \cap B) \cup A) \cap ((A \cap B) \cup \overline{B})\\
                &= (A \cup A) \cap (B \cup A) \cap (A \cup \overline{B})\\
                &= A \cap (B \cup A) \cap (A \cup \overline{B})\\
                &= A \cap ((B \cap A) \cup (B \cap \overline{B}) \cup (A \cap A) \cup (A \cap \overline{B}))\\
                &= A \cap ((B \cap A) \cup \emptyset \cup A \cup (A \cap \overline{B}))\\
                &= A \cap ((A \cap (B \cup \overline{B})) \cup A)\\
                &= A \cap A = A 
            \end{align*}
        \end{enumerate}
    \subsection*{Exercise 35}
        Let \(A, B, C\) be sets. Use the identities in Table 1 to show that \(\overline{A \cup B} \cap \overline{B \cup C} \cap \overline{A \cup C} = \overline{A} \cap \overline{B} \cap \overline{C}\)
    \subsubsection*{Solution}
        Applying De Morgan law, we have that:
            \begin{align*}
                \overline{A \cup B} &= \overline{A} \cap \overline{B}\\
                \overline{B \cup C} &= \overline{B} \cap \overline{C}\\
                \overline{A \cup C} &= \overline{A} \cap \overline{C}\\
                \overline{A \cup B} \cap \overline{B \cup C} \cap \overline{A \cup C} &= \overline{A} \cap \overline{B} \cap \overline{B} \cap \overline{C} \cap \overline{A} \cap \overline{C}\\
                &= \overline{A} \cap \overline{B} \cap \overline{C}
            \end{align*}
    \subsection*{Exercise 36}
        Prove or disprove that for all sets A, B, and C, we have
        \begin{enumerate} [label = (\alph*)]
            \item \(A \times (B \cup C) = (A \times B) \cup (A \times C)\)
            \item \(A \times (B \cap C) = (A \times B) \cap (A \times C)\)
        \end{enumerate}
    \subsubsection*{Solution}
        \begin{enumerate} [label = (\alph*)]
            \item Let \(x, y\) an ordered pair of a Cartesian Product of 2 sets. \(A \times (B \cup C)\) create ordered pair
            (x, y) with \(x \in A\) and \(y \in B \lor y \in C\). In \((A \times B) \cup (A \times C)\), we have that 
            \(A \times B\) create a ordered pair \(x \in A \land y \in B\), \(A \times C\) create a ordered pair
            \(x \in A \land y \in C\). Represent in propositions, we have that:
            \begin{align*}
                (x \in A) \land (y \in C) \lor (x \in A )\land (y \in C)\\
                = (x \in A) \land (y \in C \lor y \in B)
            \end{align*}
            Which is the same propositions as \(A \times (B \cup C)\).
            \item Let \(x, y\) an ordered pair of a Cartesian Product of 2 sets. \(A \times (B \cap C)\) create
            (x, y) with \(x \in A \land (y \in B \land y \in C)\). In \((A \times B) \cap (A \times C)\), we have that
            \(A \times B\) create an ordered pair \(x \in A \land y \in B\), \(A \times C\) create an ordered pair
            (x, y) with \(x \in A \land x \in C\). Represent in propositions, we have that:
            \begin{align*}
                x \in A \land y \in B \land x \in A \land y \in C\\
                = (x \in A) \land (y \in B \land y \in C)
            \end{align*} 
            Which is the same propositions as \(A \times (B \cap C)\).
        \end{enumerate}
    \subsection*{Exercise 37}
        Prove or disprove that for all sets A, B, and C, we have
        \begin{enumerate} [label = (\alph*)]
            \item \(A \times (B - C) = (A \times B) - (A \times C)\)
            \item \(\overline{A} \times \overline{B \cup C} = \overline{A \times (B \cup C)}\)
        \end{enumerate}
    \subsubsection*{Solution}
        \begin{enumerate} [label = (\alph*)]
            \item In propositions, \(A \times (B - C)\) can be represented in \((x \in A) \land (y \in B \land y \notin C)\).
            In \(A \times B\), it is \(x \in A \land y \in B\), \(A \times C\) is \(x \in A \land y \in C\). Therefore,
            if we take the \textit{difference}, it will be:
            \begin{align*}
                &(x \in A \land y \in B) \land \lnot(x \in A \land y \in C)\\
                = &(x \in A \land y \in B) \land (x \notin A \lor y \notin C)\\
                = &(x \in A \land x \notin A \land y \in B) \lor (x \in A \land y \in B \land y \notin C)\\
                = & \emptyset \lor (x \in A \land y \in B \land y \notin C)\\
                = & (x \in A) \land (y \in B \land y \notin C).
            \end{align*}
        This propositions is the same as \(A \times (B - C)\).
            \item We have that:
                \begin{align*}
                    \overline{A} \times \overline{B \cup C} &= \overline{A} \times (\overline{B} \cap \overline{C})\\
                    &= x \notin A \land (y \notin B \land y \notin C)\\
                    \overline{A \times (B \cup C)} &= \lnot(x \in A \land (y \in B \lor y \in C))\\
                    &= x \notin A \lor (y \notin B \land y \notin C)\\
                \end{align*}
            We see that the difference is the \(\land\) and \(\lor\) operator behind \(x \notin A\). Therefore,
            \begin{align*}
                \overline{A} \times \overline{B \cup C} \neq \overline{A \times (B \cup C)}
            \end{align*}
        \end{enumerate}
    \section*{Section 2.3}
    \subsection*{Exercise 25}
    Let \(f:\mathbb{R}\to\mathbb{R}\) and let \(f(x) > 0\) for all \(x \in R\). Show that \(f(x)\) is strictly decreasing if and only if the function \(g(x) = 1/f(x)\) is strictly increasing.
    \subsubsection*{Solution}
        We need to prove that \(f(x)\) is strictly decreasing \(\to g(x) = 1/f(x)\) is strictly increasing and conversely.
        \begin{itemize}
            \item Suppose \(f(x)\) is strictly decreasing. When it strictly decreasing, we know that \(f(x) > f(y)\), whenever \(x < y\). And we take these 2 \(x\) and \(y\) to 
            be \(f(x)\) and \(f(y)\) in \(g(x)\). \(g(x) = 1/f(x)\), \(g(y) = 1/f(y)\). Because \(f(x) > f(y)\) so that \(g(x) = 1/f(x) < g(y) = 1/f(y)\) so that \(g(x)\) is stricly increasing.
            \item Suppose \(g(x)\) is strictly increasing. When it strictly increasing, we know that \(g(x) < g(y)\), whenever \(x < y\). We have that \(g(x) = 1/f(x) < g(y) = 1/f(y)\). Because
            \(1/f(x) < 1/f(y)\) so that \(f(x) > f(y)\) and \(x > y\). Therefore, \(f(x)\) is strictly decreasing.
        \end{itemize}
        Because both cases are \textbf{true} so that \(f(x)\) is strictly decreasing if and only if the function \(g(x) = 1/f(x)\) is strictly increasing.
    \subsection*{Exercise 26}
        \begin{enumerate} [label = (\alph*)]
            \item Prove that a strictly increasing function from \(\mathbb{R}\) to itself is one-to-one.
            \item Give an example of an increasing function from \(\mathbb{R}\) to itself is not one-to-one.
        \end{enumerate}
    \subsubsection*{Solution}
        \begin{enumerate} [label = (\alph*)]
            \item In a strictly increasing function, we have that \(f(x) < f(y)\), whenever \(x < y\). From this clue, we know that \(f(x) \neq f(y)\) and \(x \neq y\).
            According to definition of one-to-one function, the function is one-to-one if and only if \(f(a) \notin f(b)\) whenever \(a \neq b\), which is the same as the inference
            got from the strictly increasing function. Therefore, a strictly increasing function from \(\mathbb{R}\) to itself is one-to-one.
            \item The function \(f(x) = x^2\) is not one-to-one function because there are 2 values \(x\) and \(-x\) have the same image.
        \end{enumerate}
    \subsection*{Exercise 27}
        \begin{enumerate} [label = (\alph*)]
            \item Prove that a strictly decreasing function from \(\mathbb{R}\) to itself is one-to-one.
            \item Give an example of a decreasing function from \(\mathbb{R}\) to itselft that is not one-to-one.
        \end{enumerate}
    \subsubsection*{Solution}
        \begin{enumerate} [label = (\alph*)]
            \item In a strictly increasing function, we have that \(f(x) > f(y)\), whenever \(x < y\). From this clue, we know that \(f(x) \neq f(y)\) and \(x \neq y\).
            According to definition of one-to-one function, the function is one-to-one if and only if \(f(a) \notin f(b)\) whenever \(a \neq b\), which is the same as the inference
            got from the strictly increasing function. Therefore, a strictly increasing function from \(\mathbb{R}\) to itself is one-to-one.
            \item The function \(f(x) = x^2 - x\) is not one-to-one function because there are 2 values \(x = 1\) and \(x = 0\) have the same image \(y = 0\). 
        \end{enumerate}
    \subsection*{Exercise 33}
        Suppose that \(g\) is a function from \(A\) to \(B\) and \(f\) is a function from \(B\) to \(C\)
        \begin{enumerate} [label = (\alph*)]
            \item Show that if both \(f\) and \(g\) are one-to-one functions, then \(f \circ g\) is also one-to-one.
            \item Show that if both \(f\) and \(g\) are onto functions, then \(f \circ g\) is also onto.
        \end{enumerate}
    \subsubsection*{Solution}
        \begin{enumerate} [label = (\alph*)]
            \item \(g\) is an one-to-one function so that \(g(a) \neq g(b)(g(a),g(b) \in B)\) whenever \(a \neq b(a,b \in A)\). \(f\) is an one-to-one
            function so that \(f(a) \neq f(b)(f(a),f(b) \in C)\) whenever \(a \neq b(a,b \in B)\). Therefore,
            \(f \circ g(a) \neq f \circ g(b) = f(g(a)) \neq f(g(b))\) because \(f\) is one-to-one function so that \(f(g(a)) \neq f(g(b))\) whenever \(g(a) \neq g(b)\) and because
            \(g\) is one-to-one function that \(g(a) \neq g(b)\) whenever \(a \neq b\) that \(f \circ g\) is one-to-one function(\(f(g(a)) \neq f(g(b))\) whenever \(a \neq b\)) when \(f\) and \(g\) are both one-to-one functions.
            \item \(f\) is an onto function so that for any \(c \in C\), there is a \(b \in B\) to \(f(b) = c\). Similar to \(g\), for any \(b \in B\), there is an \(a \in A\) to \(g(a) = b\). Therefore,
            \((f \circ g)(a) = f(g(a)) = f(b) = c\). So for every \(c \in C\), there is an \(a \in A\) to \(f(g(a)) = f(b) = c\) so that \(f \circ g\) is onto.
        \end{enumerate}
\end{document} 